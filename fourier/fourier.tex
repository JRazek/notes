%LaTeX Template for short student reports.
% Citations should be in bibtex format and go in references.bib
\documentclass[a4paper, 11pt]{article}
\usepackage[top=3cm, bottom=3cm, left = 2cm, right = 2cm]{geometry} 
\geometry{a4paper} 
\usepackage[utf8]{inputenc}
\usepackage{textcomp}
\usepackage{graphicx} 
\usepackage{amsmath,amsfonts,amssymb,amsthm}  
\usepackage{bm}  
\usepackage[backend=bibtex,style=numeric]{biblatex}  %backend=biber is 'better'
\usepackage[bookmarks,colorlinks,breaklinks]{hyperref}  
%\hypersetup{linkcolor=black,citecolor=black,filecolor=black,urlcolor=black} % black links, for printed output
\usepackage{memhfixc} 
\usepackage{pdfsync}  
\usepackage{fancyhdr}
\usepackage{array}
\usepackage[T1]{fontenc}
\usepackage{booktabs, multirow}
\usepackage[
singlelinecheck=false % <-- important
]{caption}
\usepackage{url}
\newcommand{\Lagr}{\mathcal{L}}
\newcommand{\Real}{\mathbb{R}}
\newcommand{\Complex}{\mathbb{C}}
\newcommand{\Natural}{\mathbb{N}}
\newcommand{\tangentvector}[2]{\frac{\partial #1}{\partial #2}}

\graphicspath{ {./images/} }

\pagestyle{fancy}

\renewcommand{\contentsname}{Chapter Contents}
\renewcommand{\figurename}{Figure}
\newtheorem{theorem}{Theorem}
\theoremstyle{definition}
\newtheorem{definition}{Definition}[section]
\newtheorem{example}{Example}[section]

\captionsetup[table]{name=Table}

\title{Fourier Analysis}
%\date{}

\addbibresource{references.bib}

\begin{document}
\maketitle
\hypersetup{linkcolor=black}
\tableofcontents

\section{Cauchy Riemann}
Let $f: \Omega \to \Complex$ be a complex-valued function of a complex variable with $f(z) = u(z) + i v(z)$ and $g: \Real^2 \to \Real^2$ is a function defined as $g(x, y) = (u'(x, y), v'(x, y))^T$, with $u(z) = u'(x, y), v(z) = v'(x, y)$.
Jacobian matrix of $g$ is given by
\begin{equation}
	J_g = \begin{bmatrix}
		\tangentvector{u'}{x} & \tangentvector{u'}{y} \\
		\tangentvector{v'}{x} & \tangentvector{v'}{y}.
	\end{bmatrix}
\end{equation}
The property we would like to achieve is that we somehow treat an $\Real^2$ vector as a scalar in $\Complex$, so that linear transformation given by $J_g$ is orthogonal (with appropriate scalar). Multiplication of 2 complex numbers may only scale/rotate coordinates given by the $(Re(z), Im(z))$ and so we want our matrix multiplication to follow that same property. Finally what we get is:
\begin{equation}
	J_g = \begin{bmatrix}
		\tangentvector{u'}{x} & \tangentvector{u'}{y} \\
		\tangentvector{v'}{x} & \tangentvector{v'}{y}.
	\end{bmatrix}
	= \begin{bmatrix}
		a & -b \\
		b & a.
	\end{bmatrix}
\end{equation}
And so, finally, we get the Cauchy-Riemann equations:
\begin{equation}
	\tangentvector{u'}{y} + \tangentvector{v'}{x} = 0, \\
	\tangentvector{u'}{x} - \tangentvector{v'}{y} = 0.
\end{equation}
These are the conditions under which we say that $f$ is complex differentiable at some $z_0$. Moreover, if $f$ is complex differentiable in its domain, then we say that $f$ is holomorphic.

\section{Bounded variation}

Let $f: [a, b] \to \Complex$ be a complex-valued function of a real variable. We say that $f$ is of bounded variation if the quantity
\begin{equation}
	V_a^b(f) = \sup_P \sum_{i=1}^n_A |f(x_i) - f(x_{i-1})|,
\end{equation}
where ${P \in \{A = \{x_0, x_1, \dots, x_{n_A}\}: \text{ - partition of (a, b) } \}}$.

\end{document}
